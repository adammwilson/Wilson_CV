\input{resumeformatting.tex}
\nonstopmode

\setitemize{noitemsep,topsep=2pt,parsep=2pt,partopsep=2pt}

\newcommand*{\doi}[1]{\href{http://dx.doi.org/#1}{doi: #1}}



%%%%%%%%%%%%%%%%%%%%%%%%% Begin CV Document %%%%%%%%%%%%%%%%%%%%%%%%%%%%

\begin{document}
\makeheading{ Adam~Michael~Wilson, Ph.D.}

\thispagestyle{empty}

\begin{wrapfigure}[5]{r}{.2\textwidth}
  \vspace{-20pt}
  \includegraphics[width=.18\textwidth]{AdamWilson_Mug.jpg}
\end{wrapfigure}

\section{Contact Information}


    \href{http://www.eeb.yale.edu}{Department of Ecology and Evolutionary Biology} \\
    \href{http://yale.edu}{Yale University}          \\          
    165 Prospect St, New Haven, CT 06520 USA\\
    fax:    +1 203-432-2374
    mobile: +1 240 979-7404            \\ \\
    \email{adam.wilson@yale.edu}  \hspace{20} \href{https://twitter.com/planetflux}{@planetflux}\\
    \href{http://adamwilson.us}{adamwilson.us} \hspace{55}
    \href{http://orcid.org/0000-0003-3362-7806}{\textsc{orcid}: 0000-0003-3362-7806}

    %____________________________________________________________________________________
    % Research Interests
    % \HRule 
    % \section{Research\\Interests}%
    % I am interested in the global change ecology: the study of the interactions of species at
    % the ecosystem scale within the changing  biosphere. Currently I am
    % working on a project to  understand the potential impacts of
    % climate change on the \textit{fynbos} of  the Cape Floristic Region of South Africa.

\HRule
    %____________________________________________________________________________________
    % Education

    \section{Education}
\begin{innerlist}

\item[]{\bf  Ph.D., \href{www.eeb.uconn.edu}{Ecology and Evolutionary Biology}} \hfill {\bf
  2006--2012} \\
{\sl \href{http://www.uconn.edu}{University of Connecticut}, Storrs, CT}\\
Committee: John A. Silander (UConn), Daniel Civco (UConn), Alan Gelfand (Duke), Gene Likens (Cary Institute)\\
 Dissertation: \textit{Fire and Climate: the implications of global change in the
  Cape Floristic Region of South Africa}.  \\


\item[]{\bf M.S., \href{http://www.unh.edu/esci/}{Earth Science}} \hfill {\bf{2001--2003}} \\
{\sl  \href{http://www.unh.edu}{University of New Hampshire}, Durham, NH}\\
Advisor: Cameron P. Wake\\
  Concentrating in Geochemical Systems and Climate Change \\
Thesis: \textit{Air Quality, Weather, and Respiratory Visits to the Emergency Room in Portland, Maine and Manchester, New Hampshire}\\

\item[]{\bf B.S., \href{http://biology.unh.edu/}{Biology}}, \emph{Summa Cum Laude} \hfill {\bf{1996--2000}}\\
  {\sl \href{http://www.unh.edu}{University of New Hampshire}, Durham, NH}\\
  Concentrating in Evolutionary Biology and Ecology

\end{innerlist}

\HRule\\  %____________________________________________________________________________________
\section{Professional \\ Experience}
\begin{innerlist}
\item[]{\bf
  \href{http://climate.yale.edu/news/yale-climate-energy-institute-postdoctoral-fellowship}{Climate
    \& Energy Institute Postdoctoral Research Fellow}}
\hfill {\bf 2012--Present}\\
{\sl \href{http://www.eeb.yale.edu//}{Department of Ecology and Evolutionary Biology}\\
  \href{http://www.yale.edu/}{Yale University}, New Haven, CT, USA}\\
%Research topic: \textit{Climate Change Impacts on the Vibrancy of Fall Foliage
%in the Northeastern U.S.}\\

\item[]{\bf Environment and Organisms Working Group Member} \hfill {\bf 2012--Present}\\
{\sl National Center for Ecological Synthesis \& Analysis (NCEAS) \\
  Santa Barbara, CA, USA}\\
\href{https://groups.nceas.ucsb.edu/environment-and-organisms/}{Project 12504}: 
Choosing (and making available) the right environmental layers for modeling how the environment controls the distribution and abundance of organisms\\

\item[]{\bf Postdoctoral Research Fellow} \hspace{10pt}   Advisor:
  Dr. Walter Jetz \hfill {\bf 2012--Present}\\
{\sl Department of Ecology and Evolutionary Biology \\ Yale
  University, New Haven, CT, USA}\\
NASA Climate and Biological Response project: \textit{Integrating global species distributions, remote sensing information
  and climate station data to assess recent biodiversity response to climate change} \\

\item[]{\bf NASA Graduate Research Fellow} \hfill {\bf 2009--2012}\\
{\sl Department of Ecology and Evolutionary Biology \\ 
University of Connecticut, Storrs CT, USA}\\
Research title: \textit{Fire, phenology, and weather: implications of climate change in Mediterranean ecosystems}  \\

\newpage
\item[]{\bf Graduate Research Assistant} \hspace{10pt}   Advisor: Dr. John A. Silander \hfill {\bf 2006--2009}\\
{\sl Department of Ecology and Evolutionary Biology, \\ University of
  Connecticut, Storrs, CT, USA}\\
NSF funded project: \textit{Spatio-Temporal Models of Species
Distributions and Biodiversity at High Resolution - Integrating Climate and Population Responses}\\
%Duties included: GIS analysis, field work, meteorological data
%analysis, and statistical modeling to investigate the impact of
%climate on plant populations in South Africa.\\

\item[]{\bf Community Forestry Agent, U.S. Peace Corps}
    \hfill {\bf 2004--2006} \\
    {\sl Moroccan Department of Water and Forests, \\ Arganeraie
      Biosphere Reserve, Morocco}\\
Developed ethno--botanical guide to the floral biodiversity of Amsittene Mountain Site of
Biological and  Ecological Interest and assisted local association in
the  construction of a women's educational center.\\

 \item[]{\bf Greenhouse Gas Emissions Modeler} \hfill {\bf 2000--2004}\\
{\sl Office of Sustainability Programs, University of New Hampshire \& \\ 
\href{http://www.cleanair--coolplanet.org}{Clean Air -- Cool Planet} \\ Portsmouth, NH, USA } \\
Developed a protocol and toolkit to model greenhouse gas emissions from universities.
Used this method to inventory emissions from the University of New Hampshire. Toolkit
is now used at over 1,200 universities around the
country.\\

\item[]{\bf Project Coordinator} \hspace{10pt}  Advisor: Dr. Cameron P. Wake \hfill {\bf2002--2003}\\
{\sl Integrated Human Health and Air Quality Research (INHALE) project
  \\ Durham, NH, USA}\\
Coordinated various stakeholders, grant writing, and outreach for \$300,000 research project investigating the impact of poor air quality
and weather on  human health in Northern New England.\\

\item[]{\bf Research Assistant} \hspace{10pt}  Advisor: Dr. Cameron P. Wake \hfill {\bf 2001--2002}\\
{\sl Climate Change Research Center, University of New Hampshire \\
  Durham, NH, USA}\\
Compiled and analyzed historical data (temp, precipitation, snowfall, sea level, lake ice, etc.) on climate change from New England in the past 100 years.\\


\item[]{\bf Research Assistant} \hspace{10pt} Advisor: Dr. Marilyn Walker\hfill {\bf 2000}\\
{\sl Toolik Lake Long-term Ecological Research Station \\ Institute of Arctic Biology, Alaska}\\
Assisted with phenological studies to understand the potential impacts of climate change on the tundra plant communities at a remote station in the arctic.\\

\item[]{\bf Lab Assistant} \hspace{10pt} Advisor: Dr. John Aber \hfill {\bf 1996--2000}\\
{\sl Complex Systems Research Center, University of New Hampshire}\\
Assisted in sample processing, sample analysis, fieldwork, and GIS analysis for Harvard Forest Long-Term Ecological Research Site.\\
\end{innerlist}

\newpage
\HRule \\  %____________________________________________________________________________________
\section{Grants \& \\ Fellowships \\ \bigskip \bigskip Total awards \\ to date: {\bf \$241,500}} 

\begin{innerlist}

%\item[]{\bf Total awards to date: \$237,500}\\

\item[]{\bf Foreign Research Fellowship ($\approx$\$4,000) \hfill 2014}\\
{\sl South African National Research Council, Cape Town, South Africa} 
}\\

\item[]{\bf Yale Postdoctoral Fellowship (\$110,000) \hfill 2012--2014}\\
{\sl Yale Climate and Energy Institute, Yale University, New Haven,
  CT, USA }  \\
%Research project: \textit{Climate Change Impacts on the Vibrancy of Fall Foliage
%in the Northeastern U.S.} \\

\item[] {\bf DISCCRS Fellow ($\approx$\$2,000)} \hfill {\bf 2013}\\
{\sl DISsertations initiative for the advancement of Climate Change ReSearch}\\
Travel grant to attend week-long retreat/workshop in Colorado Springs, Colorado\\

\item[]{\bf NASA Earth Science Graduate Fellowship  (\$75,000) \hfill 2009--2012 }\\
  {\sl National Aeronautics and Space Administration \& \\ University of Connecticut, Storrs, CT, USA}\\
Research project: \textit{Fire, phenology, and weather: implications of climate change in
Mediterranean ecosystems}\\

\item[]{\bf UCONN-CESE Research Award (\$8,000) \hfill 2010} \\
{\sl Center for Environmental Sciences and Engineering Multidisciplinary
Environmental Research Awards, University of
Connecticut, Storrs, CT}\\
Research project entitled: \textit{Community
 composition through space and time: developing models of vegetation dynamics}
\\

\item[]{\bf NASA Graduate Fellowship (\$7,500) \hfill 2008--2009} \\
{\sl Connecticut NASA Space Grant}  \\

\item[]{\bf Outstanding Scholar Fellowship (\$30,000) \hfill 2006--2009}\\
{\sl Graduate School, University of Connecticut, Storrs, CT} \\

\item[]{\bf UCONN-CESE Research Award (\$5,000) \hfill 2007 }\\
{\sl Center for Environmental Sciences and Engineering Multidisciplinary
Environmental Research Awards, University of
Connecticut, Storrs, CT}\\
Research project entitled: \textit{Multi-spectral Exploration of the Cape Floristic Region of South Africa
Fire, Stress, and Species recognition}
\\
\end{innerlist}

{\bf Proposal Contributions:}
\begin{innerlist}
\item[]{\bf NSF Grant DEB-1046328 (\$3 Million) \hfill 2010 }\\
{\sl University of Connecticut, Storrs, CT, USA}\\
Co-author of climate change and disturbance section and source of
preliminary data for successful NSF grant application by
Carl Schlichting entitled  {\sl Dimensions of Biodiversity: Parallel
evolutionary radiations in Protea and Pelargonium in the Greater Cape
Floristic Region}. \\

\item[]{\bf USDA grant \#0213933 (\$545,000)\hfill 2008 }\\
{\sl University of Connecticut, Storrs, CT, USA}\\
Source of preliminary data and analysis for successful grant proposal by
J. A. Silander entitled: \textit{A multi-scale approach to the forecast of potential distributions of
invasive plant species} \\

\end{innerlist}

\newpage


\HRule \\ %____________________________________________________________________________________
    \section{Peer Reviewed Publications}
 \textbf{Citation Metrics} \\
 {\small See \url{http://goo.gl/of6GUH} for updates\\}
%\begin{wrapfigure}[5]{r}{.55\textwidth}
 % \vspace{-8pt}
 % \includegraphics[width=.55\textwidth]{citations/citations.pdf}
%\end{wrapfigure}
\begin{itemize}
\item[{\bf 25}] publications
\item[{\bf 7}] first author publications
\item[{\bf 572}] total citations
\item[{\bf 13}] h-index
\end{itemize} %\bigskip%\bigskip


{\center \bf Published (and in press)}

\begin{etaremune}[label=\textbf{P\#.\arabic*}]

\item Parmentier, B.,  McGill, B., {\bf Wilson, A.M.},  Regetz, J., Jetz, W.,  Guralnick, R., Tuanmu, M., Robinson, N., Schildhauer, M. Using multi-timescale methods and satellite derived land surface temperature for the interpolation of daily maximum air temperature in Oregon. Conditionally Accepted, \textit{International Journal of Climatology} \doi{10.1002/joc.4251}

\item Parmentier, B.,  McGill, B., {\bf Wilson, A.M.},  Regetz, J., Jetz, W.,  Guralnick, R., Tuanmu, M., Robinson, N., Schildhauer, M. (2014) An assessment of methods and remotely sensed covariates for regional predictions of 1 km daily maximum air temperature. \textit{Remote Sensing} 6(9):8639-8670 \doi{10.3390/rs6098639}.

\item Xie, Y., Ahmed, K.F.,  Allen, J. M., {\bf Wilson, A. M.},  Silander, J.A., (2015) Land surface phenology and climate variation: green-up of deciduous forest communities of northeastern North America. \textit{Landscape Ecology} 30:109�123 \doi{10.1007/s10980-014-0099-7}.

\item  {\bf Wilson, A. M.}, Parmentier, B. \& Jetz, W.  (2014) Systematic landcover bias in Collection 5 MODIS cloud mask and derived products -- a global overview.  \textit{Remote Sensing of the Environment} 141:149--154. \doi{10.1016/j.rse.2013.10.025}

\item  {\bf Wilson, A. M.} \& Silander, J. A.  (2014). Estimating Uncertainty in Daily Weather Interpolations: a Bayesian framework for developing climate surfaces.  \textit{International Journal of Climatology} 34(8):2573--2584 \doi{10.1002/joc.3859}

\item Merow, C., Latimer, A. {\bf Wilson, A. M.}, McMahon, S., Rebelo, T., Silander Jr., J. A. (2014).  On using Integral Projection Models to generate demographically driven predictions of species’ distributions: development and validation using sparse data.  \textit{Ecography}. \doi{10.1111/ecog.00839}

\item {\bf Wilson, A. M.}, Parmentier, B., Jetz, W. (2014). ``Global 1km MODIS Cloud Mask Processing Path". Dataset \#820938. Supplement to: Wilson, Adam M; Parmentier, Benoit; Walter, Jetz (2014): Systematic Landcover Bias in Collection 5 MODIS Cloud Mask and Derived Products - a Global Overview. \textit{Remote Sensing of Environment}. October 23. \url{http://doi.pangaea.de/10.1594/PANGAEA.820938}.

\item  Keil, P., {\bf Wilson, A. M.}, Jetz, W.  (2014). Uncertainty, priors, autocorrelation and disparate data in downscaling of species distributions  \textit{Diversity and Distributions}. 20(7):797--812 \doi{10.1111/ddi.12199}

\item Allen, J. M., Terres, M. A.,  Katsuki, T., Iwamoto, K., Kobori, H.,  Higuchi, H., Primack, R. B., {\bf Wilson, A. M.}, Gelfand, A., \& Silander, J. A.  (2014).   Modeling daily flowering probabilities: expected impact of climate change on Japanese cherry phenology  \textit{Global Change Biology} 20(4):1251--1263 \doi{10.1111/gcb.12364}

\item Keil, P., Belmaker, J., {\bf Wilson, A. M.}, Unitt, P., \&
  Jetz, W. (2013). Downscaling of species distribution models: a hierarchical
  approach. \textit{Methods in Ecology and Evolution} 4:82--94. \doi{10.1111/j.2041-210x.2012.00264.x}

\item Jiang, X.,  Dey, D. K., Prunier, R., {\bf Wilson, A. M.},
 \& Holsinger,  K. E. (2013). A New Class of Flexible Link Function with Application to Spatially Correlated Species Co-occurrence in Cape Floristic Region.
  \textit{Annals of Applied Statistics} 7(4):1837-2457.

\item Ahmed, K. F., Wang, G., Silander, J. A., {\bf Wilson, A. M.},
Allen, J. M., Horton, R., \&  Anyah, R. (2013) Statistical downscaling and bias correction of climate model outputs for climate change impact assessment in the U.S. northeast. \textit{Global Planetary Change}, 100:320--332.  \doi{10.1016/j.gloplacha.2012.11.003}

\item {\bf Wilson, A. M.}, Silander, Jr. J. A., Gelfand, A. E., \& Glenn,
  J. (2011).  Scaling up: linking  field data and remote sensing
with a hierarchical model. \textit{International Journal of
  Geographical Information Science}, 25(3):509--521. \\
  \doi{10.1080/13658816.2010.522779}

\item De Klerk, H. M., {\bf Wilson, A. M.}, Steenkamp, K., \& Tsela, P. (2011) Evaluation of satellite-derived
Burned Area products for the Fynbos, a Mediterranean
shrubland. \textit{International Journal of Wildland Fire},
21(1)36--47. \doi{10.1071/WF11002}

\item Chakraborty, A.,  Gelfand, A. E.,  {\bf Wilson, A. M.}, Latimer,
A. M., \& Silander, Jr. J. A. (2011). Point Pattern Modeling for Degraded Presence-Only Data over Large
Regions. \textit{Journal of the Royal Statistical Society, Series C:
  Applied Statistics}, 60(5):1--20. \doi{10.1111/j.1467-9876.2011.00769.x}

\item Merow, C., LaFleur, N., Silander Jr. J. A., {\bf Wilson, A. M.}, \& 
Rubega, M. (2011). Predicting bird-mediated spread of invasive plants across
northeastern North America. \textit{American Naturalist}, 178(1):30--43.  \doi{10.1086/660295}

\item {\bf Wilson, A. M.}, Latimer, A. M., Silander, Jr. J. A., Gelfand, A. E. \& de
Klerk, H. (2010). A Hierarchical Bayesian model of wildfire in a Mediterranean
biodiversity hotspot: Implications of weather variability and global
circulation. \textit{Ecological Modelling}, 221:106--112. \doi{10.1016/j.ecolmodel.2009.09.016}

\item Chakraborty A., Gelfand, A. E., {\bf Wilson, A. M.}, Latimer, A. M.,
\& Silander, Jr. J. A. (2010). Modeling large scale species abundance with latent spatial
processes. \textit{The Annals of Applied Statistics}, 4(3):1403--1429. 
\\ \doi{10.1214/10-AOAS335}

\item  Ib\'a\~nez, I., Silander, Jr. J. A., Allen, J.,  Treanor, S., \& {\bf Wilson, A. M..} (2009).
 Identifying hotspots for plant invasions and forecasting focal
 points of further spread. \textit{Journal of Applied Ecology}, 46:1219--1228. 
 \\ \doi{10.1111/j.1365-2664.2009.01736.x}

\item  Ib\'a\~nez, I., Silander, Jr. J. A., {\bf Wilson, A. M.},
  LaFleur, N., Tanaka, N., \& Tsuyama, I. (2009). Multivariate forecasts of potential distributions
 of invasive plant species. \textit{Ecological Applications}, 19(2):359--375. \doi{10.1890/07-2095.1}

\item Primack, R. B., Ib\'a\~nez, I., Higuchi, H., Lee, S. D.,
  Miller-Rushing, A. J., {\bf Wilson, A. M.}, \& Silander, Jr, J. A.  (2009). Spatial and interspecific
variability in phenological responses to warming temperatures. \textit{Biological
Conservation}, 142(11):2569--2577. \doi{10.1016/j.biocon.2009.06.003}

\item {\bf Wilson, A. M.}, Wake, C. P., Kelly, T., \& Salloway, J. C. (2005). Air
pollution, weather and respiratory emergency room visits in two
northern New England cities: an ecological time-series
study. \textit{Environmental Research}, 97:312--321. \doi{10.1016/j.envres.2004.07.010}

\item Keim, B. D., Fischer, M. R., \& {\bf Wilson, A. M.} (2005). Are there
spurious precipitation trends in the United States Climate Division
database?. \textit{Geophysical Research Letters}, 32:L04702. \doi{10.1029/2004GL021985}

\item {\bf Wilson, A. M.}, Salloway, J. C., Wake, C., \& Kelly, T. (2004). Air Pollution and
the Demand for Hospital Services: A Review. \textit{Environment International},
30:1109--1118. \doi{10.1016/j.envint.2004.01.004}

\item Keim, B. D., {\bf Wilson, A. M.}, Wake, C. P., \& Huntington, T.
  G. (2003).
Are there spurious temperature trends in the United States Climate
Division database? \textit{Geophysical Research Letters}, 30(7):1404. \doi{10.1029/2002GL016295}

\end{etaremune}

%\newpage 
{\bf In review \& in preparation (including target journal)}

\begin{etaremune}

\item {\bf Wilson, A. M.}, Latimer, A.M., Silander, J. A. Climatic controls on ecosystem resilience: implications of a changing climate on post-fire regeneration in the Cape Floristic Region. In revision, \textit{PNAS}.

\item Slingsby, J., {\bf Wilson, A. M.} Fire, biodiversity and climate change: a missing link in Fynbos science.  In review, \textit{South African Journal of Science}.

\item {\bf Wilson, A. M.}, Likens, G. E. Content volatility of scientific topics in Wikipedia: A Cautionary Tale. In review, \textit{PLOS One}.

\item {\bf Wilson, A. M.}, Jetz., W. High-resolution global cloud dynamics for ecosystem and biodiversity monitoring.  In preparation, \textit{Nature}.

\item Domisch. S., {\bf Wilson, A. M.}, Jetz, W., Integrating multiple data types into freshwater species distribution models. In preparation, \textit{Methods in Ecology and Evolution}.

% wikipedia 
% protea demography


\end{etaremune}


%\newpage
\HRule %____________________________________________________________________________________

    \section{Invited \\ Seminars \& Keynotes}
    
    \begin{etaremune}

	\item {\bf Wilson, A. M.} (2014, November). Department of Geography, University at Buffalo, NY, USA. 

	\item {\bf Wilson, A. M.} (2014, November). Tentative title: \textit{Species to ecosystems: integrating satellite and field data to understand resilience in a biodiversity hotspot}. 
	Berkeley Initiative in Global Change Biology, University of California Berkeley, CA %\\ %\doi{10.6084/m9.figshare.1142275}

	\item {\bf Wilson, A. M.} (2014, August). \textit{Climatic controls on ecosystem resilience: combining hierarchical modelling with space borne monitoring of past fire plant biomass accumulation}. 
	Keynote Address, Fynbos Forum, Knysna, South Africa.\\  \doi{10.6084/m9.figshare.1142275}

	\item {\bf Wilson, A. M.} (2014, May). \textit{Ecosystem dynamics: disturbance and recovery in the Cape Floristic Region of South Africa}. Department of Ecology and Evolutionary Biology. University of Connecticut, Storrs, CT. \\ \doi{10.6084/m9.figshare.1025883}
	    
	\item {\bf Wilson, A. M.} (2014, February). \textit{From imperfection to inference: issues of scale and uncertainty in global change biology.} Department of Environmental Science, Policy, and Management Colloquium. UC Berkeley, Berkeley, CA. \\ \doi{10.6084/m9.figshare.947682}

	\item {\bf Wilson, A. M.} (2011, June). \textit{Weather data for phenological
    analysis}. International Workshop on climate change and phenology, Boston University, Boston, MA.

	\item {\bf Wilson, A. M.}. (2010, August). \textit{Climate Change and Fynbos: Fire, Growth, and Survival}.
 Climate Systems Analysis Group, University of Cape Town, South Africa.

\item {\bf Wilson, A. M.}. (2009, July). \textit{Climate Change, Fire, \& Biomass from Space}. South African National Biodiversity Institute, Cape Town, South Africa.

\end{etaremune}
\HRule

    \section{Selected conference presentations \\ \&
      posters}


\begin{etaremune}
\item Jetz, W.,  {\bf Wilson, A. M.}, Tuanmu, M,  Melton, F., Guzman, A., Parmentier, B., McGill, B.J., Guralnick, R.,  Amatulli, G., (2015, May) \textit{Development and use of a new suite of global, remote sensing based environmental layers for biodiversity monitoring and prediction}.   36$^{th}$ International Symposium on Remote Sensing of Environment (ISRSE), Berlin, Germany. \url{http://www.isrse36.org}

\item Jetz, W., Keil, P., {\bf Wilson, A. M.}, O�Hara, R.B., Mertes, K. and Domisch, S., (2014) \textit{Integrating Processes and Data Types for Predicting Species Distributions across Spatial Scales.} Presented at the 99th ESA Annual Meeting, Sacramento, CA, August 15. \url{http://eco.confex.com/eco/2014/webprogram/Paper49492.html}

\item {\bf Wilson, A. M.}, \&  Jetz, W., (2014) \textit{High-Resolution Cloud Climatology for Global Land Areas.} Poster presented at the NASA Biodiversity and Ecological Forecasting Team Meeting, Silver Spring, MD, May 8, 2014. \\ \url{https://www.signup4.net/public/ap.aspx?EID=20142567E&OID=50}

\item Latimer, A. M., {\bf Wilson, A.M.,} \& Merow, C. (2014) \textit{Using Statistical Models to Study Climate-Disturbance-Plant Interactions.} Presentation presented at the SAMSI Program on Mathematical and Statistical Ecology, Durham, NC, August. \url{http://goo.gl/NzDhwL}.

\item Xie Y., Allen J. M., {\bf Wilson, A.M.,} Silander J. A. (2013) \textit{Land Surface Phenology and Climate Variation:  Green-up of Deciduous Forest Communities of Northeastern North America.} Institute of Botany,  Chinese Academy of Sciences Invited Seminar, Beijing, China. 

\item {\bf Wilson, A. M.}, Parmentier, B., McGill, B., Guralnick, R.,
  \&  Jetz, W. (2013, January). \textit{Incorporating Satellite
    Derived Cloud Climatologies to Improve High Resolution
    Interpolation of Daily Precipitation}. Poster presented at the 6th
  International Conference of the International Biogeography Society, Miami, FL.

\item Moses, K., Noell, N.,  Casado, D.,  Rijal, R.,  Medina, Y.,  Lewis, L.,  Mendez, M., Caballero, P., Morales, V., {\bf Wilson, A. M.},  Vezzani, P., Massardo, F., Sancho, L., Russel, S., Cavieres, L. A., Goffinet, B., Rozzi, R. (2013) \textit{Ecotourism with a Hand Lens in the Miniature Forests of Cape Horn: A Sustainable Pathway for Bryophyte Conservation.} In Life on Earth: Preserving, Utilizing, and Sustaining Our Ecosystems. Minnesota, USA: Ecological Society of America.

\item Parmentier, B., McGill, B. Regetz, J., {\bf Wilson, A. M.},
  Jetz, W., Guralnick, R., Schildhauer, M., \& Narro, M. (2013, January).
  \textit{Climate Interpolation of Daily Maximum Temperature:
    Improvements for the Production of Climate Datasets.} Poster
  presented at the 6th
  International Conference of the International Biogeography
  Society, Miami, FL.

\item Keil, P., {\bf Wilson, A. M.}, Belmaker, J., \&  Jetz, W. (2013, January).
  \textit{Downscaling of geographical distributions of individual species and species richness}. Poster
  presented at the 6$^{th}$  International Conference of the International Biogeography
  Society, Miami, FL.

\item  {\bf Wilson, A. M.}, Silander Jr., J. A., \& Latimer, A. M.  (2012, October). \textit{Climatic controls on
    ecosystem resilience: Post-fire regeneration in the Cape Floristic
    Region of South Africa}. Selected speaker at the RCN FORECAST New Investigators Conference:
New perspectives on data assimilation in global change science,
Woods Hole, MA.

\item Latimer A. M., {\bf Wilson, A. M.}, \& Silander, Jr. J. A. (2012, October).
  \textit{ Using data from different scales to model plant population 
responses}. Presented at the RCN FORECAST New Investigators Conference:
New perspectives on data assimilation in global change science,
Woods Hole, MA.

\item  Keil, P., {\bf Wilson, A. M.}, \& Jetz, W.  (2012, October).
  \textit{Combining data of different spatial resolutions to predict
    species' distributions at fine grain}. Poster presented at the RCN
  Forecast New Investigators Conference: New perspectives on data assimilation in global change science in
Woods Hole, MA.

\item Allen J. M., Katsuki, T., Iwamoto, K., Kobori, H., {\bf Wilson, A. M.},
  \& Silander, Jr. J. A. (2012). \textit{Japanese Cherry Flowering Responses to
    Projected Climate Change}. Presented at Phenology 2012 Conference in
  Milwaukee, WI.

\item Latimer A. M., Merow, C., \& {\bf Wilson, A. M.}. (2012, August).
  \textit{Hierarchical statistical models for ecological data:
    Combining explanation and prediction}. Presented at the 97$^{th}$
  Annual Meeting of the Ecological Society of America, Portland, Oregon.

\item  {\bf Wilson A. M.}, Silander Jr. J. A., \& Latimer, A. M.  (2012, August). \textit{Climatic controls on
    ecosystem resilience: Post-fire regeneration in the Cape Floristic
    Region of South Africa} (\#36482). Presented at the 97$^{th}$
  Annual Meeting of the Ecological Society of America, Portland, Oregon.

\item Kilroy, H.A., {\bf A. M. Wilson}, C. Merow, \& J.A. Silander
  Jr. (2012, July)
A new method of estimating fynbos plant community composition via
remote sensing, presented at Fynbos Forum, Cape St Francis, South Africa.

\item Allen, J. M., Silander Jr, J. {\bf Wilson, A. M.}, Primack, R. B., Kobori, H., \& Katsuki, T. (2011). \textit{Springtime phenological responses in a survival analysis framework}. COS2 - Phenology. presented at the 96$^{th}$ Annual Meeting of the Ecological Society of America, Austin, Texas.

\item de Klerk, H., {\bf Wilson, A. M.}, \& Steenkamp,
  K. (2010). \textit{Evaluation of satellite-derived burned area
    products for the Fynbos, a Mediterranean shrubland}.  Presented at
  the 5$^{th}$ International Wildland Fire Conference, Sun City, South Africa.

\item {\bf Wilson, A. M.}, Silander, Jr. J. A., Gelfand, A. (2010, April). \textit{Understanding Fire and Climate in Mediterranean
Ecosystems: an Integrated Approach}.  Land Cover Land Use Change Science
Team Meeting, Bethesda, MD.


\item Primack, R.B., Ib\'a\~nez, I., Higuchi, H., Lee, S. D.,
  Miller-Rushing, A.J., {\bf Wilson, A. M.}, \&  Silander,
  Jr. J. A. (2009, August). \textit{Forecasting trends in species 
    phenological responses to global warming: The predictive potential of
    multi-site data}. Presented at the 94$^{th}$ Annual Meeting of the
  Ecological Society of America, Albuquerque, NM.

\item Allen, J., Ib\'a\~nez, I., {\bf Wilson, A. M.}, Treanor, S. A.,
  \& Silander, Jr. J. A. (2009, April). \textit{Identifying hot spots of plant species invasions and assessing
foci of further spread}. Presented at the Odum Conference titled Understanding and
managing biological invasions as dynamic processes: Integrating
information across space and time.  E.N. Huyck Preserve \& Biological
Research Station, Rensselaerville, NY. 

\item Belcon, A., Comita, L., Isbell, F.,  Linares, R., Rojas, C., \& {\bf
    Wilson, A. M.} (2009, May).
\textit{Facilitating Global Change Research in the Tropics: Science and Data
Management}.  Presented at the Global Change and Tropical Ecosystems
Course, Organization for Tropical Studies \&
Pan-American Advanced Studies Institute, La Selva Biological Reserve, Costa Rica.

\item {\bf Wilson A. M.} (2009, April). \textit{Climate Change, Fire, \& Biomass from
Space}.  Presented to the Global Change and Tropical Ecosystems
Course, Organization for Tropical Studies \& Pan-American
Advanced Studies Institute, La Selva Biological Reserve, Costa Rica.

\item {\bf Wilson, A. M.} (2008, September). \textit{Implications of Climate Change in Mediterranean
Ecosystems: Modeling Fire Dynamics}. NASA Northeast Regional Space
Grant Meeting,  Windsor Locks, CT.

\item {\bf Wilson, A. M.} (2008, May). \textit{Monitoring Wildfire
    from Space}. Presented at the NSF-USDA International Workshop on Supercomputing
  Applications in Climate Sciences and Remote Sensing, Cairo, Egypt. 

\item {\bf Wilson, A. M.}, Latimer, A. M., Silander, Jr. J. A. (2007, October). \textit{The Fire-Weather
Relationship in the South African Fynbos: Implications under Climate
Change}. Poster presented at the Integrative Graduate Education and Research Traineeship
(IGERT) Conference on Sustainability to Understand Social-ecological
systems, Fairbanks, Alaska.

\item Ib\'{a}\~{n}ez, I., Silander, Jr. J. A., {\bf Wilson, A. M.},
  \& Lafleur, N.
(2007). \textit{Challenges of modeling invasive species
  spread}. Presented at the Ecological Society of America Joint Meeting, San Jose, California.


\item Ib\'{a}\~{n}ez, I., Silander, Jr. J. A., {\bf  Wilson, A. M.},
  \& Lafleur, N. (2007). \textit{Modeling
patterns of future plant invasions in the New England
region. Colonization versus invasion: do the same traits matter?},
Presented to the Federal Institute of Technology, Ascona, Switzerland.

\item Ib\'{a}\~{n}ez, I., Silander, Jr. J. A., {\bf Wilson, A. M.},
  \& Lafleur, N. (2007, September). \textit{Challenges
of modeling invasive species spread}. Seminar presented to the Department of
Ecology, Evolution, and Environmental Biology, Columbia
University, New York, New York.

\item Ib\'{a}\~{n}ez, I., Silander, Jr. J. A., {\bf Wilson, A. M.},
  \& Lafleur, N. (2007, April). \textit{Modeling
patterns of future plant invasions in New England}. Harvard Forest
Seminar Series, Harvard University, Cambridge, MA.

\item Latimer, A. M., {\bf Wilson, A. M.}, \& Silander, Jr., J. A. (2007). \textit{Linking
changing climate, productivity, and fire in the Cape Floristic Region:
A spatio-temporal Bayesian analysis of fire frequency}. Presented at
the ESA/SER Joint Meeting, San Jose, California.

\item LaFleur, N., Ib\'{a}\~{n}ez, I., Silander, Jr. J. A.,  Mehrhoff,
  L., \&   {\bf Wilson, A. M.} (2007). \textit{Modeling patterns of future plant invasions in the New England
region}. Presented at the Connecticut Conference on Natural Resources, Storrs, CT.

\item LaFleur, N., Ib\'{a}\~{n}ez, I., Silander, J. A., Mehrhoff, L., \&
  {\bf Wilson, A. M.}
(2007, February). \textit{Modeling patterns of future plant invasions in the New England
region}. Presented at the Weed Science Society of America Meeting, San Antonio,
Texas.

\item {\bf Wilson, A. M.}, Latimer, A. M.,  Silander, Jr. J. A. (2007). \textit{The Fire-Weather
Relationship in the South African Fynbos: Implications under Climate
Change (\#5608)}. Presented at the Society for Conservation Biology Meeting, Port
Elizabeth, South Africa.


\item Wake, C. \& {\bf Wilson, A. M.} (2004). \textit{Multiple Indicators of Climate Change
Over the Past Century in New England} (\#A42A-04). Presented at the AGU Fall meeting, San Francisco,
December.

\end{etaremune}

\HRule\\  %____________________________________________________________________________________

\section{ Teaching \& Mentoring} 
{\bf Courses}
\begin{itemize}
\item[]{\bf Methods in Spatial Biodiversity Analysis} \hfill {\bf 2013 \& 2015}\\
   {\sl Yale University, Ecology \& Evolutionary Biology (EEB 713)}\\
Co-developed and taught new course with three sections: 1) methods and tools (Linux; command line scripting; GRASS; advanced R for spatial/environmental data) 2) example \emph{big} datasets (environmental, remote sensing, biodiversity), and 3) example questions, data integration and biostatistics (data-model dichotomy, Bayesian approaches, addressing uncertainty).\\

\item[] {\bf{Statistics in the Life Sciences (\textit{Guest Lecturer})}}  \hfill {\bf Fall 2014}\\
   {\sl Yale University, Ecology \& Evolutionary Biology (EEB 210)}\\
Statistical and probabilistic analysis of biological problems presented with a unified foundation in basic statistical theory. Problems are drawn from genetics, ecology, epidemiology, and bioinformatics. \\

\item[]{\bf Geospatial Cyberinfrastructure: Climate Modeling} \hfill {\bf Fall 2011}\\
   {\sl University of Connecticut, Ecology \& Evolutionary Biology (EEB 5894)}\\
Co-designed short course to expose graduate students to the science of climate
change and global change biology, focusing on accessing and evaluating
climate summary and weather station data from
various sources. We examined 1) the basics of the climate system
2) how and why our climate is changing, and 3) how to use that
knowledge to understand and predict ecological processes.\\


\item[]{\bf An Introduction to \texttt{R} Programming}
\hfill {\bf Spring 2008}\\
   {\sl  University of Connecticut, Ecology \& Evolutionary Biology (EEB 5894)}\\
Designed and co-taught this course to provide an introduction to the \texttt{R} language for
graduate and advanced undergraduate students.  We covered basic
programming principles for ecological modeling and statistics.  \\

\end{itemize}
   
\item[]{\bf Supervision and Mentoring of Student Research} \hfill {\bf 2008--Present}\\
   \begin{itemize}
   \item Ben Carlson (Graduate, Yale University, 2014): \textit{Environmental annotation of large biodiversity databases}
   \item William Freedberg (Undergraduate, Yale University, 2013): \textit{Global meteorological station gap analysis}. 
   \item Katherine Morrow (Undergraduate, University of Connecticut, 2011): \textit{Interspecific variation in fall leaf color}. 
   \item Colin Carlson (Undergraduate, University of Connecticut, 2011): \textit{Phenotypic plasticity and extinction risk in South African plants: a reaction norm approach to ecological modeling}.
   \item Adam Pellegrini (Undergraduate, University of Connecticut, 2009): \textit{Role of topographic variation and micro-climate in driving fine-grain variability of biomass in fynbos shrubland ecosystems}.
   \item John Glenn (Undergraduate, University of Connecticut, 2008): \textit{Development of fynbos biomass sampling methods}
   \end{itemize}\\ \bigskip
      
%      \newpage
      
      {\bf Workshops}
\begin{itemize}
\item[] {\bf{Using spatial biodiversity data and remote sensing to support conservation decision-making}}  
\hfill {\bf 2015}\\
International Society of Tropical Foresters, Yale University, New Haven, CT

\item[] {\bf{Integrative spatial biodiversity analysis and Map of Life}}  \hfill {\bf 2015}\\
International Biogeographic Society 7$^{th}$ Biennial Meeting, Bayreuth, Germany

\item[] {\bf{Geo-spatial and environmental analysis on open-source software}}  \hfill {\bf 2014}\\
Program in Spatial Biodiversity Science \& Conservation, Yale University, New Haven, CT, USA

\item[] {\bf{Ecosystem Disturbance and Resilience}}  \hfill {\bf 2014}\\
South African National Biodiversity Institute (SANBI) \& University of Cape Town, South Africa.

\item[] {\bf{Biodiversity Informatics Workshop}}  \hfill {\bf 2013}\\
International Biogeographic Society 6$^{th}$ Biennial Meeting, Miami, FL

\item[] {\bf{   Graduate Research Symposium committee}} \hfill {\bf 2006--2011}\\
{\sl  Department of Ecology and Evolutionary Biology,  University of
  Connecticut} \\
Organized an annual symposium for graduate students to present
their research.
\end{itemize}
\bigskip

{\bf Other Educational Activities \& Experiences}

\begin{itemize}

      \item[]{\bf Classroom activity contributor} \hfill {\bf 2014}\\
Developed materials for website accompanying the textbook: Shuster, M., Vigna, J., Sinha, G., & Tontonoz, M. (2014). \textit{Scientific American Biology for a Changing World} 2nd Edition. Macmillan Higher Education. \\


\item[]{\bf Yale Scientific Teaching Fellowship} \hfill {\bf Fall 2012}\\
   {\sl Yale University}\\
Selected through competitive application process for a program designed by Joan Handelsman (President Obama's Associate Director for Science) introducing `scientific teaching,' an evidence-based approach to STEM education built on engaging students in active learning,
leveraging diversity, and assessing student learning. Designed,
taught, and revised instructional materials for an introductory level
biology course. \\


\item[]{\bf UConn International Research Experience to South Africa \hfill 2008--2010}\\
   {\sl University of Connecticut}\\
Served as a mentor to undergraduate students as they developed independent
projects and completed four months of field work in South Africa (over
three trips). One of my manuscripts was developed from this collaboration.\\



\item[]{\bf Environmental Education \hfill 2004--2006}\\
   {\sl Department of Water and Forests \\ Arganeraie Biosphere Reserve,
   Morocco}\\
Served as an environmental educator in the U.S. Peace Corps. The recent establishment of the Arganeraie
Biosphere Reserve limited how villagers could use the surrounding
forest and I worked with them to seek alternative and sustainable
livelihoods that did not degrade their environment.\\

\item[]{\bf Outdoor Education \hfill 1996--present}\\
   {\sl Various locations in USA, Canada, and South Africa}\\
Served in various outdoor education positions over the last 15
years including the following: Wilderness Guide (1997--1999) at the Maine High
Adventure Area, Trip Leader (1996--2000) for the New Hampshire Outing
Club,  Trip Leader (2003) for Camp Tree Tops in Lake Placid, NY leading a five-week
backpacking/kayaking/caving trip in the Canadian Rockies. Recently I've led and assisted with international research
expeditions (such as the IRES program mentioned above) and been responsible for safety, environmental education,
and navigation for the group in remote areas.\\

\end{itemize}

%\newpage
\HRule\\  %____________________________________________________________________________________
    \section{Service}
{\bf   Reviewer for:}
\begin{itemize}
\item \textit{University of Chicago Press}
\item \textit{Biology Letters (Royal Society)}
\item \textit{Global Ecology and Biogeography}
\item \textit{Remote Sensing}
\item \textit{Bioscience}
\item \textit{Ecography}
\item \textit{Oecologia}
\item \textit{ISPRS Journal of Photogrammetry and Remote Sensing}
\item \textit{International Journal of Climatology}
\item \textit{Remote Sensing of Environment}
\item \textit{Journal of the Royal Statistical Society: Series C}
\end{itemize}


{\bf   Member of:}
\begin{itemize}
\item Ecological Society of America
\item International Biogeography Society
\item Union of Concerned Scientists
\item North American Nature Photographers Association
\end{itemize}\\
\medskip

\HRule \\ 
%____________________________________________________________________________________
    \section{Software \& Computational Skills}

\textbf{Software Expertise:}\\
R (statistical computing and graphics), LINUX/UNIX, SQL, NetCDF
Operators (NCO), Climate Data Operators (CDO), Bayesian
Gibbs Samplers (OpenBUGS \& JAGS), \LaTeX, Drupal CMS, GRASS GIS, Quantum GIS, cluster
computing (\textit{e.g.} NASA's Pleiades Supercomputer), Inkscape
Vector Graphics, Adobe Lightroom Professional Photo Editing Software.\\

Public code repository available at \url{https://github.com/adammwilson}.\\

{\bf R Packages}\\
\begin{itemize}
\item {\bf  Wilson, A.M.},  (2014). \textit{rasterAutocorr: Quickly calculate spatial autocorrelation on 2D rasters}. R package v0.9. \url{https://github.com/adammwilson/rasterAutocorr}.\\
\item Vieilledent, G.,  Latimer, A. M.,  Gelfand, A. E.,  Merow, C.,  {\bf  Wilson, A.M.},   Mortier, F.,  &  Silander 
Jr., J. A.. (2014). \textit{hSDM: hierarchical Bayesian species distribution models.} R package v1.4. \url{http://CRAN.R-project.org/package=hSDM} 
\end{itemize}

%\newpage

\HRule \\ %____________________________________________________________________________________
    \section{Nature Photography}
%Biodiversity conservation needs a compelling story to captivate a
%broad audience. Maps and graphs are critically important, but require context to
%engage non-scientific audiences. To communicate effectively about scientific issues, one must both
%“arouse and fulfill” the audience. Scientific details can be
%fulfilling, but are seldom arousing to non-scientists.
%Photography, on the other hand, is a powerful medium to captivate an
%audience while sharing scientific knowledge.

{\bf Books}\\

Rozzi, R., Lewis, L.,  Massardo, F.,  Medina, Y.,  Moses, K., Mendez,
M., Sancho, L., Vezzani, P.,  Russell, S., \& Goffinet, B., {\bf Photographs by 
Wilson, A. M.} (2012). \textit{Ecotourism con Lupa en el Parque Omora}. Sub-Antarctic Biocultural Conservation Program, Santiago, Chile.\\

Goffinet, B., Rozzi, R.,  Lewis, L., Buck, W., \& Massardo, F. {\bf Photographs by Wilson, A. M.} (2012). \textit{Miniature Forests of Cape Horn:  Ecotourism with a Hand Lens}. University of North Texas Press, Denton, TX, USA.\\

{\bf Photographic publications, awards, presentations, and exhibitions}\\

%Photographic contribution to: Lewis, L. \& Goffinet, B. (2012, June)
%\textit{Long distance hitchhiker or relict? Phylogenetic pattern in the Tetraplodon mnioides complex
%(Splachnaceae) and implications for the origin of T. fuegianus.}
%Presented to the 3$^{rd}$ International Symposium on Molecular Systematics
%of Bryophytes. \\

{\bf Photo Exhibitions \hfill Fall 2010}\\
Photograph (\textit{Leaf Cutter Ants}) displayed at the Phipps Conservatory and Botanical Gardens, Pittsburgh, PA. \\

{\bf Third Place,  Eco-Photo contest \hfill July 2010}\\
Ecological Society of America\\

{\bf College Photography Scholarship (\$1,500) \hfill February 2010}\\
North American Nature Photographers Association\\

{\bf Winning photograph for the \textit{Naturally Funny} competition \hfill
February 2010 }\\
North American Nature Photographers Association\\

{\bf Photographic Contributions \hfill 2009--2010} \\
Over 100 photographs donated to the
\href{http://www.ots.duke.edu/}{Organization for Tropical Studies}
(OTS) for use in 
promotional literature, calendars, and newsletters.  \\

{\bf Conservation Photography Activities \& Service} \\

{\bf College Committee Member \hfill 2010--present }\\
{\sl North American Nature Photographers Association (NANPA)} \\
Duties included judging applicants and organizing their
conservation photography project at the annual summit\\

{\bf Co-organizer of \textit{Connecting Through the Lens} \hfill March 2011}\\
Co-organizer of an intensive two-day conservation photography workshop funded by the Legacy Institute for
Nature \& Culture held in McAllen, Texas.  The workshop empowered photographers to use their craft to connect
people to nature, translate science, and facilitate difficult conversations. \\

{\bf  Multi-media mentor \hfill March--April 2011 }\\
Mentor for the production of a short film, \textit{Reconnecting the
  Rio Grande  Valley}, developed by the winners of the 2011 NANPA
college scholarship.  The film is currently being shown by the U.S. Fish and Wildlife Service. \url{http://vimeo.com/21488710}\\

%\newpage


{\bf Project Photographer \hfill January 2011 }\\
Field Photographer for a National Science Foundation funded project
(\textit{Integrating ecological sciences and environmental philosophy
  for bio-cultural conservation in the temperate and subantarctic
  Ecoregions of southern South America}) in the Chilean Patagonia. 
Products included photographic contributions to: outreach and educational materials for the Omora
Ethnobotanical Park, articles in the \textit{North Texas Daily newspaper} (2/9/2011),  \textit{La
Prensa Austral} (a Chilean newspaper, 2/21/2011),  the
\textit{Giornale Di Vicenza} (an Italian newspaper, January 9, 13, 14,
15, 2011), and \textit{el Mercuio} (a Chilean newspaper, 11/13/2012) and the BBC (3/2/2015).  \\

%\newpage

\HRule\\  %____________________________________________________________________________________
    \section{Outreach, \\ press, reports, \\ non-peer reviewed \\
      publications \& \\ software} 

{\bf Yale Climate \& Energy blog \hfill 2012--Present}\\
Regular contributor of \textit{perspective} articles and news  at  \url{http://climate.yale.edu}\\ receives $\approx$3,000 views per month.\\

{\bf PlanetFlux blog \hfill 2009--Present}\\
My blog (\url{http://adamwilson.us/planetflux}) \textit{PlanetFlux: Musings on the Science of Global Change, Remote Sensing,
Statistical Computing, Scientific Visualization, and more...}
receives $\approx$250 views per month.\\

%{\bf ``Best'' Personal Webpage \hfill 2009--2010}\\
%{\sl Department of Ecology and Evolutionary Biology at the University of
%Connecticut}\\
%My site (\url{http://adamwilson.us}) was twice voted the best personal
%webpage in the department and currently receives $\approx$300 hits/month\\

{\bf Featured graduate student \hfill 2010}\\
{\sl University of Connecticut President's Annual Report}\\
\textit{\href{http://uconn.edu/par/2010/index.php}{What's your spark:
    Graduate  Research at the University of Connecticut}}\\


Bejbouji, J., {\bf Wilson, A. M.}, Hmaidouch, A. (2006). \textit{Contribution \a
  l'\'etude
de la biodiversit\'e floristique du Jbel Amsitten: Inventaire et
utilisation des plantes (Contribution to the study of Floral
Biodiversity of Mount Amsitten: An Inventory and Utilization of
Plants)}. Special Report of the Ministry for the Protection of Water
and Forests of the Kingdom of Morocco (in French).\\

{\bf Indicators of Climate Change in the Northeast \hfill 2005}\\
{\sl \href{http://www.cleanair-coolplanet.org}{Clean Air - Cool Planet} Special Report} \\

{\bf Wilson, A. M.} \textit{Campus Carbon Calculator\texttrademark}. Clean Air-Cool
Planet Special Publication. This tool is used at over 1,200
universities across the country to estimate their greenhouse gas
emissions.  It has become the ``tool of record'' for most of the 600
signatories to the \textit{American Colleges and University Presidents Climate Commitment}.\\

%\newpage
\HRule  \\%____________________________________________________________________________________
    \section{Training \& \\ Skills} 

\textbf{Workshops / Internships / Short Courses:}
\begin{enumerate}

\item[] {\bf DISsertations initiative for the advancement}  \hfill {\bf 2013}\\
{\bf of Climate Change ReSearch (Fellow)}\\
La Foret Conference Center, Colorado Springs, CO, \url{http://disccrs.org/}

\item[] {\bf Georeferencing Workshop (Participant)}  \hfill {\bf 2012}\\
Society for the Preservation of Natural History Collections, Yale University

\item[] {\bf Global Change and Tropical Ecosystems (Participant)}  \hfill {\bf 2010}\\
Attended six week workshop offered by the Organization for Tropical Studies,
Duke University, and the Pan-American Advanced Studies Institute in Costa Rica

\item[] {\bf Mapping Invasive Plants for IPANE (Participant)}  \hfill {\bf 2008}\\
Attended training for the Invasive Plant Atlas of New England, University of Connecticut

\item[] {\bf Internship in Sustainable Development (Participant)} \hfill {\bf{1999}}\\
Auroville, Tamil Nadu, India
\end {enumerate}

%\newpage 

\textbf{First Aid:}\\
 Certified in Wilderness First Aid, SOLO-New
          Hampshire (2003), Certified in CPR for the Professional
          Rescuer (2003), Emergency Medical Technician
          (EMT-Basic) (2000)\\

\textbf{Additional Languages:}\\
 Tashelheit (\textit{alias} Berber): the language of
the indigenous people of Morocco.\\

\textbf{Travel Experience}:\\
 To conduct my dissertation
research, conduct workshops, and for other reasons, I have visited the
following locations outside the Continental U.S. (for the time
specified): Morocco (twenty seven months), South Africa (ten months over six trips), India
(four months), Alaskan Arctic (three months), Costa Rica (six weeks),
Canadian Rockies (five weeks), Chile (four
weeks), France (two weeks), Egypt (two
weeks), Ireland (two weeks), Lesotho (two days), and Turkey (one day).\\

%\HRule  %____________________________________________________________________________________
%    \section{\mysidestyle Current \& Past Collaborators} 
%\begin{itemize}
%\item Alan Gelfand
%\item Walter Jetz
%\item Gene Likens
%\end{itemize}
% Jasper 

\newpage
%\HRule\\  %____________________________________________________________________________________
    \section{References} 
%\begin{multicols}{2}
%\setlength{\columnseprule}{0.05mm}

\textbf{John A. Silander} \\
Professor \\
Department of Ecology and Evolutionary Biology \\
University of Connecticut\\ 
75 N. Eagleville Road, Unit 3043 \\
Storrs, CT  06269-3043 \\	
860-486-2168 \\
\url{http://www.eeb.uconn.edu/people/silander/}\\
\href{mailto:john.silander@uconn.edu}{john.silander@uconn.edu}

\medskip 

\textbf{Walter Jetz} \\
Associate Professor \\
Ecology and Evolutionary Biology \\
Yale University\\ 
165 Prospect St.\\
New Haven, CT 06520, USA \\	
203-432-7540 \\
\url{http://jetzlab.yale.edu/}\\
\href{mailto:walter.jetz@yale.edu}{walter.jetz@yale.edu}
\medskip
%\columnbreak

\textbf{Gene E. Likens} \\
Distinguished Senior Scientist \\
Founding Director \& President Emeritus \\
Institute of Ecosystem Studies \\ 
2801 Sharon Turnpike\\
P.O. Box AB \\
Millbrook, NY  12545-0129 \\
845-677-5343 \\
\url{http://www.caryinstitute.org/science-program/our-scientists/dr-gene-e-likens}\\
\href{mailto:likensg@caryinstitute.org}{likensg@caryinstitute.org}
%\end{multicols}

\medskip 

\textbf{Alan E. Gelfand} \\
Chair and J.B. Duke Professor \\
Statistics and Decision Sciences \\
Duke University \\ 
223-A Old Chemistry Building\\
Duke University \\
Durham, NC 27708-0251 \\	
919-668-5229 \\
\url{http://www.isds.duke.edu/~alan/}\\
\href{mailto:alan@stat.duke.edu}{alan@stat.duke.edu}

\medskip

\textbf{Andrew M. Latimer} \\
 Associate Professor\\
Department of Plant Sciences University of California \\ 
One Shields Avenue  \\
Davis, CA 95616 \\
530-752-0896 \\
\url{http://www.plantsciences.ucdavis.edu/faculty/latimer}\\
\href{mailto:amlatimer@ucdavis.edu}{amlatimer@ucdavis.edu}



%________________________________________________________________________________________

\end{document}

%________________________________________________________________________________________
% EOF

